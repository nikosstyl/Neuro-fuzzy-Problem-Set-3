% !TeX spellcheck = en_US
\section{Problem 8}
The term "ordinary subset of level $\alpha$ in fuzzy set theory refers to a non-fuzzy subset that is created from a fuzzy subset by taking into consideration only those elements whose degrees of membership are at least $\alpha$. In essence, it's a threshold-based selection from a fuzzy set in which each element is only included if its membership score is greater than or equal to $\alpha$.\\
Conversely, this idea is extended to pairs of components for a fuzzy relation by the "ordinary relation of level $\alpha$". A set of ordered pairs with corresponding degrees of membership that indicate the strength of the relationship between the elements form a fuzzy relation. Thus, the set of all pairs whose membership in the fuzzy relation is at least $\alpha$ is the ordinary relation of level $\alpha$. \\

Analytically, the ordinary relation of level $\alpha$ for a fuzzy relation is defined as the set of all pairs $(x,y)$ for which the membership function $\mu_{\tilde{R}}(x,y)$ is greater than or equal to $\alpha$. Alternatively, it is a crisp set that is derived from the fuzzy relation by incorporating all element pairs with a degree of membership greater than the specified level $\alpha$\\

For a fuzzy relation with a membership function $\mu_{\tilde{R}}(x,y) = 1 - \frac{1}{1+x^2+y^2}$, the ordinary relation of level $0.3$ is the set:\\
$R_{0.3} = {(x,y) | \mu_{\tilde{R}}(x,y) \geq 0.3}$\\
This means you are looking for all the pairs $(x,y)$ where the membership value is at least $0.3$.\\

To determine analytically the ordinary relation of level $0.3$, we will solve the inequality
\begin{align*}
	\mu_{\tilde{R}}(x,y) \geq 0.3 \\
	1 - \frac{1}{1+x^2+y^2} \geq 0.3 \\
	\frac{1}{1+x^2+y^2} \leq 0.7 \\
	1 + x^2 + y^2 \geq \frac{1}{0.7} \\
	x^2 + y^2 \geq \frac{1}{0.7} - 1
\end{align*}
From basic maths we know that $	x^2 + y^2$ is the equation of a circle. This represents the region outside a circle centered at the origin with a radius squared of $\frac{1}{0.7} - 1$. By calculating the radius we can plot the region that satisfies the ordinary relation of level 0.3.